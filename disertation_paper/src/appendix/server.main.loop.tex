\chapter{Offload Server Main Loop}
\label{appendix:servermainloop}


\lstset{language=c,caption=Main Loop of BLEOffloadingFramework server, label=lst:servermainloop}
\begin{lstlisting}
void start_main_loop_rfcomm(){
    int result = -1;
    int i = 0;
    fd_set afds;
    fd_set rfds;
    struct timeval tv;
    int clientRfSock;
    struct sockaddr_rc rem_addr = {0};
    socklen_t rfcommConnInfoLen;
    socklen_t sockAddrLen;
    bdaddr_t clientBdAddr;

    printf("Starting main loop and listening for rfcomm sockets..\n");
    while(1){
        FD_ZERO(&afds);
        FD_SET(rfcomm_socket, &afds);
        tv.tv_sec = 1;
        tv.tv_usec = 0;

        result = select(FD_SETSIZE, &afds, NULL, NULL, &tv);
        if(result < 0){
          printf("select doesn't work.\n");
          break;
        }

        for(i = 0; i < FD_SETSIZE; ++i)
          if(FD_ISSET(i, &afds)){
             if ( i == rfcomm_socket){
                /*Connection Request on original socket */
                sockAddrLen = sizeof(rem_addr);
                clientRfSock = accept(rfcomm_socket, (struct sockaddr *) &rem_addr, &sockAddrLen);
                if(clientRfSock < 0){
                  printf("Error accepting a socket."); break;}
                  FD_SET(clientRfSock, &afds);
                }
                else{
                  /* If socket is not available do not read from it or handle requests */
                  if(socket_available(i) == 0) continue;
                  
                  /*Data arriving on already connected socket*/
                  char buffer[MAX_BUFFER_SIZE];
                  
                  mutex_lock(client_mutex(i));
                  int read_bytes = read(i, buffer, MAX_BUFFER_SIZE);
                  mutex_unlock(client_mutex(i))
                  
                  printf("%s\n", buffer);
                  handle_request(i, buffer, read_bytes);
                }
          }
          
          clean_up_fds();
        }
}
\end{lstlisting}
