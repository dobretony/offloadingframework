\chapter{Framework Implementation}
\label{chapter:implementation}

With the scope of helping developers create power efficient applications through the use of offloading and to demonstrate the inherent gain of code offloading and data binding through Bluetooth Low Energy in this chapter we will present the BLEOffloadingFramework. This framework was constructed for use with the Android Operating System and provides an Application Programmable Interface (API) for developers to use inside their applications in order to offload their more CPU-intensive tasks to a BLE-capable Linux machine.


\section{The client - Android Framework}

As previously state, Android is a mobile operating system tailored specifically for smartphones and other embedded devices, such as TVs or wearables, that offers a wide range of features for users and developers alike. In order to create an application for these devices, a developer may use the Software Development Kit (SDK) \abbrev{SDK}{Software Development Kit} that is offered for every version of Android.

The BLEOffloadingFramework uses the Android SDK version 20 and upwards, as this API level has the features needed for effective Bluetooth Low Energy communication.\index{Android SDK}

Because we are discussing a system that is to be used by developers, one prime objective is to be modular: the framework must not interfere with the application and must be coupled or decoupled from the application with ease. The BLEOffloadingFramework presents itself as a Java library that developers can integrate into their application, without changing too much the actual code. This library contains a class called OffloadService\index{OffloadService}, which is an Android Service that provides:

\begin{itemize}

\item{Searching and establishing connection automatically to Offload Server}

\item{Handling of the lifecycle of the Bluetooth Low Energy socket}

\item{Notifications to the user of the smartphone when offloading is possible}

\item{An interface to developers through which they can perform offloading requests}

\end{itemize}


The offload service is present in the library offered to developers and has to be declared in the application manifest in order to be used (example in Listing \ref{lst:manifest}). As well, any application that desires to connect with the framework must declare that it needs permission for BLUETOOTH and BLUETOOTH_ADMIN. This represents all the changes needed in the configuration of the application.

\pagebreak

\lstset{language=xml,caption=Example Application Manifest,label=lst:manifest}
\begin{lstlisting}
<uses-permission android:name="android.permission.BLUETOOTH"/>
<uses-permission android:name="android.permission.BLUETOOTH_ADMIN"/>
<uses-feature android:name="android.hardware.bluetooth_le" android:required="false"/>
...
<application>
...
	<service android:name="com.offloadingframework.offloadlibrary.OffloadService"/>
...
</application>
\end{lstlisting}


In order to start the service, the application needs to perform a request called "binding" to the Android system. The developer can choose to perform this action at application start up, when the needs arises or even permit the user to select when to perform offloading through settings or other similar methods.

Once enabled, the service will start a low energy scan filtering for a specific Universally Unique Identifier (UUID) \abbrev{UUID}{Universally Unique Identifier}. This type of identifier is a standard used in software construction in order to enable distributed systems to uniquely identify information without significant central coordination. In Bluetooth Low Energy and normal Bluetooth, UUIDs are used, among other things, to uniquely determine services offered to other devices, such as the Gatt Service, HID service, etc.

The 128-bit value UUID that is used by the BLEOffloading framework\index{Offload Service UUID} is shown in Listing \ref{lst:uuid}.

\lstset{language=java,caption=UUID used for discovery of Offloading Service, label=lst:uuid}
\begin{lstlisting}
private static UUID uuid = UUID.fromString("6686416b-aa45-798f-1747-680bd1c4a59b");
\end{lstlisting}


The offload service will perform low energy scans by default for 5 seconds at 10 seconds intervals. The developer can tailor these values by overriding the "offloading_ble_scan_interval" and "offloading_ble_scan_duration" properties of the library.

Once an offloading server with the selected UUID is identified through scans, the service will interrogate the GATT service for its characteristics, identifying as much information about the server, such as its location, its time and date and its availability. If the information provided proves agreeable (such as the availability of offloading services) then the offload service creates a RFCOMM socket to the server.

The RFCOMM socket remains active as long as both devices are in range and periodically sends and receives at least a PING/ACK message. This type of message passed between the mobile phone and the server determines if the server can handle requests. If at any time an ACK message is missed the server is considered lost or out of range and the process of scanning is started anew. An example of the work the service is doing during in its initialization phase is given in Figure \ref{img:offloadserviceinit}.

\fig[scale=0.5]{src/img/OffloadServiceInit.png}{img:offloadserviceinit}{Actions started and managed by the OffloadService}


Besides managing the connection the Offload Service provides a Message Thread that uses the already established Android Handler architecture in order to receive and send messages to the application. The handler works within the same thread as the service and provides methods designed for message passing and running methods at certain points in time. The ScanningRunnable and "PING/ACK Runnable" run using this handler and its \textit{postDelayed()} method. The \textit{postDealyed()} method allows the running of certain methods after an amount of time has passed. This is very helpful when you require the same task to be executed periodically.

Another feature of the Android Handler system is message passing. Every application and service can have a Handler that runs on its main thread that listens to messages and takes certain actions when a message is received. The offload service handler uses these messages in order to provide an interface to the user application.

In the context of BLEOffloadFramework, any request to the offload server is done through message passing to the OffloadService. For the task delegation offload method mentioned in Section \ref{section:taskdelegation}, the application can send a message of a certain type, defined in the static class \textit{Contants} and register a Runnable callback in order to wait for the result. This class exposes a number of tasks that can be offloaded, with the certainty that these algorithms exist and can be run on the offload server.

For the case of Data Binding solution, there is a special message type of message called \textbf{REQUEST_GENERAL_OFFLOAD} that will deliver to the offload service an OffloadRequest object that will be sent to the server. If the server accepts these types of requests, a callback with the result will be called when offloading is done. The callback will receive a not available answer when offloading is not possible or the server does not accept that request.

As previously stated, the communication between the offload service and the server is done through a RFCOMM socket managed by the client. In order to speed up the process, all the control messages sent through the socket are of the byte primitive type. All the user data is converted to byte form before sending, in a process called encoding. The server decodes the byte string that is sent through the channel and processes the request. All the results or responses sent back to the client are also encoded. This strategy permits the encoding and decoding process to also contain data archiving steps, thus minimizing the time it takes to transfer data across the RFCOMM channel.

An example offloading call on a test application is present in appendix \ref{appendix:clientimplementation}.

\pagebreak

\section{The server - Linux Embedded System}

The offloading server has been designed with the possibility of running on multiple Linux based machines ranging from desktops to laptops or even other embedded devices such as Raspberry PI or MinnowBoard Max.

The server must be Bluetooth Low Energy compatible and uses the open source BlueZ\cite{holtmann2007bluez}\index{BlueZ} library as means for communication through Bluetooth. This library offers an API and a set of wrappers in order to manage and create Bluetooth connections, manage pairing requests and handle several Low Energy features such as advertising and scan filtering.


\subsection{Server structure}
\label{subsection:serverstructure}

The BLEOffloadingFramework server project is written in the C language and has the following structure described in Figure \ref{img:serverstructure}:

\fig[scale=0.5]{src/img/ServerStructure.png}{img:serverstructure}{The structure of the BLEOffloadingFramework Server}

\begin{itemize}

\item{Root - \textit{main.c server.c}}
	These files represent the starting point of the server program. Here, the main variables are initialized, checking is performed in order to see if Bluetooth can be enabled on the system and also contains the main loop that listens for connections and starts additional threads by request.

\item{Communication API - comm}
	In this directory there are wrappers for the BlueZ library, specifically for the starting of advertising technique with certain parameters, initialization of both LE Gatt Service and SDP service registering. This directory also contains the initialization of the RFCOMM sockets and the mutex synchronized methods for writing and reading from those sockets.

\item{Auxiliary Programs - aux}
	This directory contains the functions and programs required for the task delegation solution. Each algorithm is contained in its own file with a header file that describes it and the parameters it needs in order to be called.
	
\item{Third Party applications - third_party}
	This folder contains a set of programs that should be called when the server initializes. These represent the programs needed for the data binding of loose coupled systems offloading solution. Each program has its own sub-folder and needs to be registered in the server.c file mentioned above.
	
\item{Helper functions - other}
	In this directory there are present some additional files that are needed in order to speed up the process, such as UUID transformations and encoding/decoding helper functions.

\end{itemize}

This structure permits the organization of the server in a scalable way. Developers that want to add methods for Task delegation can do so by adding a C file and a C header file in the \textit{aux} folder. Afterwards they must add the type of request the server will send to the \textbf{handle_request()} function referenced in \textit{server.c} main file. 

For the Data Binding through loose coupling method, a developer can add their binaries in the \textit{third_party} folder and add the same \textbf{handle_request()} logic as above.

The project Makefile is created in such a way to include all the files from a specific folder and also add binaries from third parties in order to create a running executable. Additional libraries and functions that can be used as-is can be inserted inside a separate folder under \textit{third_party}.


\subsection{Server logic}
\label{subsection:serverlogic}

The server needs to be started with root privileges, as it relies on communication with the Bluetooth kernel module and needs access to the HCI device present on the machine. A special user can be created that has access to the Bluetooth subsystem and such avoid the risk of this server causing security concerns.

When it first boots up, the offload server will try to set the parameters for advertising by calling through HCI the BT device present on the machine. If this is successful, it will then activate the advertising feature on the chipset with those specific parameters. After this phase, it will register a Low Energy GATT service that describes the services provided by the server through Bluetooth and also registers a new service with the Service Discovery Protocol (SDP) \abbrev{SDP}{Service Discovery Protocol}, a protocol used by classic Bluetooth in order to identify the capabilities of devices in a piconet.

After registering these services, the initialization of the RFCOMM\index{RFCOMM} socket is performed. This socket is set in the \textit{listen} state and we use the Linux \textit{select}\cite{ostrowski2000mechanism} mechanism in order to perceive new events on that socket. The \textit{select} mechanism "allows event monitoring on multiple file descriptors, waiting until one or more of the files descriptors become ready for some class of I/O operations"\footnote{select - Linux ManPages (the output of "man select" on any Linux system)}. As such, in the main loop of the server application a \textit{select} is performed on the RFCOMM socket with a maximum number of "child" file descriptors for each subsequent client socket connected. The code behind this operation is described in appendix \ref{appendix:servermainloop}.

Noteworthy is the \textit{handle_request()} near the end of the main loop, as this where the process of decoding and selecting the request happens. Once in this stage, the server application decodes the received message and creates an appropriate response to the client.

If it can handle the request then it will send a preemptive \textbf{ACCEPTED} message to the client, in order for it to prepare for the upcoming result. If in its featured database, which is actually a parsable XML file, the method request for offload does exist, then it will send a \textbf{REJECTED} message back to the client.

After sending the \textbf{ACCEPTED} message, the server will launch a new thread that will serve as the processing environment for the offloaded task. That thread will use the thread safe socket write and read defined in \textit{comm/bluetooth.c} in order to transmit data back to the client.

In the case of loose coupled systems, the server launches during its \textit{init} phase the programs that are coupled to it. These processes have to handle their own lifecycle and have to terminate when the parent process (the offload server) sends a SIGINT process signal. When a socket is opened with the request for that specific program, the socket is passed to the child process. 

The child process must not close the socket, as it will cause a reconnect from the smartphone client, and the offload server must not use that specific socket once the request has been sent. This is done using a special mark on the RFCOMM socket that states that the socket is in use by another process. When the work is done on that socket, it will be marked as free by the child process and such the server can continue to receive other requests.

Finally, after each socket iteration, a clean up is performed on the open file descriptors. This clean up is based on the fact that a socket has received updates in the last two minutes. If a socket has not been used in more that two minutes, it will be closed, thus freeing that file descriptor for future use.

To note is the fact that the server does not take into account the type of device it receives updates from. It does not collect identification data of any kind and it acts mostly as a relay for other devices.






