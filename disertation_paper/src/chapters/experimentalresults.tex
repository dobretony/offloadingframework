\chapter{Experimental Results}
\label{chapter:results}

After defining the architecture of the BLEOffloadingFramework and how it is implemented, in this chapter we will define a testing methodology that can be used to see the benefit of using offloading over Bluetooth in applications.

First off we have to define the terms used in calculating the gain through offloading. Let \(\Delta\tau\) be the gain of power resulted from using offloading instead of normal work on the current system. \(\Delta\tau\) is expressed in energy consumption over a specific time period and is defined as the energy consumption over a certain period of time required to perform a task (\(\tau_{local}\)) on a given machine minus the energy required on the same machine to send that task over to another machine and receive the result (\(\tau_{offload}\)). As such, we can describe the gain as in Formula \ref{gainequation1}:

\begin{align}\label{gainequation1}
\delta\tau = \tau_{local} - \tau_{offloaded}
\end{align}

Both \(\tau_{local}\) and \(\tau_{offloaded}\) are experimental variables and should be expressed in the energy it took to perform a certain action over a period of time (this is the measurement of power). In order to calculate the gain in a more general fashion that does not depend on the device that it runs on, we will consider the amount of energy over time as being expressed in battery usage, a relative term used in smatphones and other similar devices to express the amount of battery power left in their devices. This variable is expressed in the percentage of energy left in the battery versus the energy contained in the battery at full capacity. In this case, \(\tau_{local}\) is expressed in Formula \ref{gainequation2} with the following relative terms:

\begin{align}\label{gainequation2}
\tau_{local} = \tfrac{Percentage\ of\ battery}{Period\ of\ time\ while\ task\ has\ run}
\end{align}

Because we expresses this in percentages, the calculations are not exact, but by repeating the task and ignoring the little overhead caused by the initialization of variables or stack calling, we can extend the period of time to a magnitude of hours. This permits us to show the larger impact upon the user with ease.

In the same style, we have to define \(\tau_{offload}\) in Formula \ref{gainequation3}.

\begin{align}\label{gainequation3}
\tau_{offload} = \tau_{BLEConnectionHandling} + \tau_{WaitPeriod} + \tau_{BLETransfer}
\end{align}

Where each term of the equation can be expressed as stated in the Formula \ref{gainequation4}, Formula \ref{gainequation5} and Formula \ref{gainequation6}.

\begin{align}\label{gainequation4}
\tau_{BLEConnectionHandling} = \tfrac{Percentage\ of\ battery}{The\ amount\ of\ time\ the\ connection\ is\ alive\ and\ is\ managed} 
\end{align}
\begin{align}\label{gainequation5}
\tau_{WaitPeriod} = \tfrac{Percentage\ of\ battery}{The\ amount\ of\ wait\ time\ for\ offloading\ to\ be\ done} 
\end{align}
\begin{align}\label{gainequation6}
\tau_{BLETransfer} = \tfrac{Percentage\ of\ battery}{The\ amount\ of\ time\ it\ takes\ for\ the\ data\ to\ be\ transferred}
\end{align}

With these notions in mind we will first explain the experimental setup with an example offloading scenario based on task delegation and afterwards determine the impact of the above variables in a comparison between methods.

\section{Framework Test Setup and Results}

In order to determine the experimental gain of the framework a test setup has been devised for the BLEOffloadFramework. 

This setup uses as a test case the example application provided in Appendix \ref{appendix:clientimplementation}. The application shows to the user an image generated with the Mandelbrot set \cite{mandelbrot2013fractals} with the set number of pixels for width and height, scaled to fit on the device screen.

The same algorithm is provided through task delegation on the offloading server and is accessible by the application through the \textbf{TASK_MANDELBROT} message type provided to the offloading service.

In this setup we will permit an external program to perform user behavior type clicks on the application through a testing process called Instrumentation \cite{kropp2010automated}. "Instrumentation testing allows you to verify a particular feature or behavior with an automated JUnit TestCase. You can launch activities and providers within an application, send key events, and make assertions about various UI elements." \footnote{\url{http://www.netmite.com/android/mydroid/development/pdk/docs/instrumentation_testing.html}} This permits us to do the same task (generate the Mandelbrot set) over a period of two hours. During this time we will measure the battery charge by registering a BroadcastReceiver for battery level and logging every two minutes.

During the period, the device will be put in a special mode, called "Airplane Mode" in which all data communication is disabled, in order to minimize the impact of the system on the test results. Bluetooth can be enabled in this mode separately and is enabled during the offload test, thus accounting for the energy consumption of the Bluetooth subsystem of Android.

We will discuss two main cases:

\begin{enumerate}
\item{Generating Mandelbrot with offloading disabled}

	This will cause a separate thread to be launched inside the application process that will generate a Mandelbrot of 1600x1024 pixels with an iteration mark of 100.
	
\item{Generating Mandelbrot with offloading enabled}

	This will cause the OffloadService to begin its activity of scanning for an offload server, connecting to it and send a request for a Mandelbrot set with the specified parameters.
	
\end{enumerate}

The results of running this setup is presented in \ref{img:comparisondata}. 

To note is the fact that although this setup was tailored to a specific test application, the setup steps remain the same for any application developed using the BLEOffloadFramework. These steps are as follows:

\begin{enumerate}
\item{Implement application with both offloading and local implementations.}

\item{Create a common user scenario that would be used inside the application.}

\item{Implement the user case into a JUnit test case using the Instrumentation Test Framework of Android.}

\item{Implement inside the test case or in a separate dummy service, a BroadcastReceiver for battery charge updates.}

\item{Run the test for a predefined amount of time that should show impact. On lower end devices, 2 hours is sufficient, while on higher devices more than 8 hours is advised.}

\item{Compare results for both offloading enabled and disabled.}
\end{enumerate}

The result of the aforementioned test setup can be expressed in a graphic of data points, expressed in the percentage of battery at a given time frame. As such Figure \ref{img:offloaddisableddata} expresses the percentage of battery over 60 samples in 2 hours when offloading is disabled. Figure \ref{img:offloadenableddata} represents the same sample size during 2 hours when offloading is disabled. Both figures have been obtained on the same device, an Intel based tablet with a 2 Ghz processor with 1Gb of RAM.

\fig[scale=0.5]{src/img/datawhenoffloadingdisabled.png}{img:offloaddisableddata}{A graphic depicting battery usage when offloading is disabled.}

\fig[scale=0.5]{src/img/datawhenoffloadingenabled.png}{img:offloadenableddata}{A graphic depicting battery usage when offloading is enabled.}

Figure \ref{img:comparisondata} represents the comparison between Figure \ref{img:offloaddisableddata} and Figure \ref{img:offloadenableddata} as the dependency of battery charge over the period of time.

\fig[scale=0.5]{src/img/comparisonofbatteryusage.png}{img:comparisondata}{Comparison between battery usage in the test setup.}

We can define the gain \(\Delta\tau\) as the difference between the battery percentage when offloading is enabled and when offloading is disabled, during the whole period of time. In Figure \ref{img:comparisondata} we can establish that even if at the beginning no difference can be recognized, while we progress in time the variation of battery charge becomes more and more feasible. This goes to prove that the offloading technique is plausible over Bluetooth Low Energy, pertaining a gain over a longer period of time.

The low difference in battery usage can be attributed to the fact that during offloading, the Bluetooth system is enabled and this provokes more CPU usage and communication between different subsystems.

Also noteworthy is the fact that during the 2 hours period, the Mandelbrot set was generated 1430 times in the case of local computation, versus 1200 times when offloading said computation to a remote server. This is because the applications standard approach was to wait for the image to be completed on the offload server and then shown on the screen through the use of the Bitmap and ImageView API of Android. This downtime has been noted as \(\tau_{WaitPeriod}\) in our previous declarations and can be mitigated when offloading through the method of data binding.

In the loose coupled model for offloading the server either has data available, in the case of caching, or it needs very few computation overall. As such the time required for waiting for data to become available is very small, which leads to an increased \(\tau_{WaitPeriod}\), thus maximizing the gain overall. To note is the fact what when doing task delegation, the local gain is higher for that specific task, even accounting for the wait period. The purpose of task delegation is to save as much battery life, even if it takes a bit longer for the task to complete.

Because of the advantage the loose coupled method brings, it also has to be considered when developers are using the BLEOffloadingFramework. 

When considering using one method or the other there are some guidelines to consider:

\begin{itemize}

\item{The object that has to be offloaded}
	
	When designing the application with offloading in mind, it is important to note the offloading target. If the target is a task that needs to be performed rarely, but it requires a lot of CPU time in order to be accomplished, it is best to use the task delegation solution in order to maximize the local offloading gain. If the target requires constant access to data, the overall gain can be higher if the loose coupled solution is used.
	
\item{Development speed}

	BLEOffloadingFramework offers a multitude of tasks and algorithms already implemented in the server. As such, it is much faster to implement offloading through task delegation than creating a new process to be run by the offload server, as is the case for loose coupled solutions.

\item{Availability of data}

	In the end it goes down to data: does the application needs unprocessed data to be fetched from the Internet and then presented to the user? does that data needs to be processed first? can it be cached for future use by other offloading clients? If the design is data centric, then the loose coupled method is probably best, while as if processing is involved, then task delegation might be the solution.

\end{itemize}