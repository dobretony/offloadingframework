\chapter{Background and Related Work}
\label{chapter:stateoftheart}

This chapter presents the technical background of the technologies used in constructing the BLEOffloadingFramework. As previously stated, Bluetooth wireless technology is used as a communication channel between offloading devices, as such the key terms of this standard are presented in Section \ref{ble}.

The main consideration of the offloading framework is extending the battery life of embedded devices, with a special focus on smartphone mobile devices. In section \ref{android} the Android smartphone operating system is presented and the motivation for choosing this type of devices as the beneficiary for the framework.

In section \ref{relatedwork} several offloading systems are presented and the main advantages of BLEOffloadingFramework over other systems is detailed.

\section{Bluetooth Low Energy}
\label{ble}

Bluetooth(BT) has been a long standing standard for small area wireless communications. Most mobile devices, ranging from PDAs to mobile phones and other gadgets, use this technology in order to communicate effortlessly over short distances, making possible file transfers, contact sharing, wireless audio and video streaming and much more. With the progress of Internet and the Cloud, though, the need for small Personal Area Networks has been reduced, as its drawbacks became more and more obvious - battery life of mobile devices has been reduced and the added overhead of Bluetooth communications is not sustainable, has a low throughput and a small range. 

Together with the specification of Bluetooth 4.0, Bluetooth SIG has also announced the standard for Bluetooth Low Energy(BLE)\cite{gomez2012overview}. This standard focuses on a trade-off between energy consumption, latency, piconet size and throughput. The advent of this standard, versus other similar wireless solutions such as ZigBee, is due to the fact that it is applicable in a larger variety of use cases: healthcare devices, small electronics, low power devices, Internet of Things or security measures.

This standard also offers full backwards compatibility, as the added benefit of low-energy transmissions can be used in parallel with the normal Bluetooth 4.0 specification. This is applicable because BLE mainly relies on parameter configuration and short, but consistent, device discovery.

In classic BT applications, when two devices needed to communicate they had to be set in Discoverable mode, identify each other and create a secure connection in a process referred to as pairing and then follow the specifications of certain Profiles. We can compare this wireless connection capability to the OSI stack, where instead of protocols, we have profiles that specify how to interact with different devices. For example, in order to connect to a Bluetooth enabled
Mouse or keyboard and use its facilities the device needs to follow the guidelines of the Human Interface Device (HID) profile. An example of the Bluetooth Low Energy stack can be seen in Figure \ref{img:bluetoothstack}.

\fig[scale=0.5]{src/img/BluetoothStack.png}{img:bluetoothstack}{Bluetooth Low Energy Stack}

Just as in classic Bluetooth\cite{haartsen2000bluetooth}, the BLE protocol stack is comprised of two main components: the Host and the Controller. The Controller component is usually a small piece of hardware, integrated on a System-on-a-Chip and contains  the Physical Layer, which controls wireless modulation, and the Link Layer. The Host component is comprised of several protocols, each working on top of each other:

\begin{itemize}

\item{Logical Link Control and Adaptation Protocol (L2CAP) - \index{L2CAP}this layer is based on the Bluetooth BR/EDR L2CAP and its prime objective is to multiplex data of higher layer protocols}

\item{Attribute Protocol (ATT)- \index{ATT}this protocol defines the communication between two devices, in a client-server architecture. On the server side, a set of data structures called attributes is maintained. The client has the ability to access and modify attributes through requests to the server.}

\item{Generic Attribute Profile (GATT) - \index{GATT}this profile determines methods through which devices can perform discovery of services and reading/writing characteristics(a set of data comprised of values and properties).}

\item{Security Manager Protocol (SMP) - \index{SMP}This profile cannot be attributed to a specific place in the stack, as it represents the security standards that BLE has to adhere to. The main security modes that are present in BLE are called LE Security Mode 1 and LE Security Mode 2, which work on two separate layers: Link Layer and ATT layer.}

\item{Generic Access Profile (GAP) - \index{GAP}This profile specifies the roles and procedures that the device has to adhere to in order to provide the discovery of services and the management of connections.}

\end{itemize}

It is worth mentioning that the GAP profile permits several operating modes, through which several techniques can be established. One such technique is called Advertising. In this technique, a device assumes the role of GAP Broadcaster, which sends small packets of data constantly. These packets contain a string of bytes that are used for identification of the device, the service used and also vendor-specific bytes.

Devices can act as Observers in BLE and such pick up on notifications and packets transmitted by a device in Broadcaster role. This technique is called scanning, and most devices can efficiently scan for advertisement packets by applying filters at the Link Layer and such only receive notifications if there is a specific Broadcaster device in range.

The BLEOffloadingFramework uses this technique in order to identify servers. If a mobile device desires to offload, it will start a scan over Bluetooth Low Energy and waits to see if it picks up any packets from a server. If such a packet exists, then the offloading framework on the client side becomes active and when an application wants to use this system a request will be generated.

When the mobile devices exceeds the server range, it will loose contact with the server (no more advertising packets detected) and such it will shutdown the framework until a new server comes in range.

Even though the main drawback would be the small range of BLE, which is around 10 meters for most devices and is dependent on the hardware, using this model of server detection and data transfer represents a tradeoff between the latency caused by a network connection to a distant cloud server in other offloading systems and the availability of such systems.

\section{Android}
\label{android}

Android\index{Android} is an operating system designed for smartphones, with a focus on usability, touch input and efficiency, both in power and computational abilities. It was first defined as a "software stack for mobile devices that includes an operating system, middleware and key applications" \cite{developers2011android}. Today, Android exists on numerous devices, including smart watches, TVs and even cars \cite{androidwiki}.

This operating system is designed on top of a modified Linux Kernel\footnote{Linux Kernel - http://www.kernel.org} with a specific stack designed with user applications on the very top. This model permits the enabling of security protocols on the lower levels (closer to the hardware), while providing a feature rich environment for third party developers that deliver content to this specific ecosystem.

Applications for the Android framework are created using a set of tools and helping methods called a Software Development Kit(SDK). Google, the maintainer of Android, offers this SDK in order for developers to create and maintain content for their smart phones. The SDK can also be augmented with an Integrated Development Environment(IDE)\abbrev{IDE}{Integrated Development Environment} that uses the tools offered in order to further simplify the work of developers.

In the top tier of the Android architecture, two main components are of use for developers especially: the Activity and the Service classes. We mention these as classes, because all applications are based on the Software Development Kit, which is available in the Java language.

The Activity\index{Android Activity} class is the main component of an Android application and represents one or more windows that the user can interact with using input methods, be it touchscreen or keyboard and mouse. Activities run in the foreground of the system, have their own life cycle  and usually occupies the whole screen, offering the user content and information. An application can contain several Activities and they are organized in a stack fashion: unless specifically closed, the Activities are pushed in a stack, so that when the user is done with the top level activity, he can immediately have access to the previous activities by pressing the back button.

The Service\index{Android Service} class represents a background process that can be created in order to offer support for applications. These classes do not have a user interface and are usually started when an Activity explicitly calls to activate the Service or when it performs an operation called Binding. Process Binding in Android represents a form of secure Inter-Process Communication(IPC) \abbrev{IPC}{Inter-Process Communication} between applications and services provided by the Android framework. The Service class is especially useful for the BLEOffloadingFramework, as the model for the offloading client is based on a background application that manages the Bluetooth Low Energy communication with the server.

Because of the availability of this code, the plethora of devices it runs on an the simple interface that it provides, the BLEOffloadingFramework is constructed mainly for the Android Operating System.


\section{Related Work}
\label{relatedwork}

This article is based primary on the works of \cite{olteanu2013extending} in which an offloading mechanism based on the application life cycle is proposed.

In this model, an application has several states such as interaction with an user via the GUI, processing multiplayer commands, simulations, graphic pipe rendering or terrain generation and some of them are done cyclically by the application. The research is based on the fact that certain states of the above loop can be offloaded completely on other devices or on servers in the cloud. One such application that fits this pattern is OpenTTD. 

An example regarding OpenTTD is the offloading of the Artificial Intelligence agents that act as players throughout a game. These agents give out commands and take decisions like a real player and are a core part of the application infrastructure, that use up a lot of processing power, depending on the complexity of the algorithms used. One way to improve on this technique is to search for other states that can be offloaded, besides the agent scripts, or to apply a fine-grain distributed technique. 

As an example we can either use the same technique in the GenerateWorld state, which is an initialization state. We can send the settings used to generate over a network communication to a cloud service and generate the world there, the result being a large amount of already processed data that the application can use. While for small worlds this method might bring a very small improvement, or none at all, for large maps can benefit from this technique. By applying fine-grain distributed technique we can mark the methods or parts of code that can be offloaded and offload them. This means that we have to create an abstraction that encapsulates the code, which is usually process-intensive, and offload that segment to a server that knows how to interpret it and simulate results. This technique resembles the Java Remote Method Invocation or Remote Procedure Calls.

The framework presented in this paper proposes a more generic approach to offloading, in that the focus will be more on optimizing the data path between devices together with the capability of applying this framework on almost any type of application.

In \cite{chun2011clonecloud} a more general offloading solution is proposed. In this method, using the high performance and mobility of cloud technology a virtual machine is created that simulates the exact environment of an application from a mobile device. With such a medium, code translating from mobile device to another machine is straightforward and less error prone. The only drawback that may occur is the fact that data usually has to be transferred across multiple points in order to be processed and that the mobile device has to always be online.

The BLEOffloadingFramework will try to handle these problems by reducing the amount of time spent on data communication, while maintaining the generic aspect of the code that can be offloaded.