\chapter{Conclusion and Future Development}
\label{chapter:conclusion}

\todo{NOTE ON FIRST DRAFT: This is a stub and will be expanded upon.}

Consuming less energy has been a prime focus in mobile communities since the development of high end smart phone devices has exceeded the technological advancement of batteries.

Offloading computational intensive code from one device to another is a method through which power saving can be achieved and the BLEOffloadingFramework is one of many solutions available to handle this problem. This system offers a complete solution for Android application developers to bring new value to their programs and to receive more out of the mobile environment. The solutions presented here are scalable, designed to be easy to use and offer a new dimension when it comes to programming for the Internet Of Things.

Improvements to the system can also bring a new form of distributed performance boost, because, even though the case for a stable server and mobile client was discussed, the framework can be made to work with other mobile devices as well, permitting users to share their power between them in a seamlessly and easy way.