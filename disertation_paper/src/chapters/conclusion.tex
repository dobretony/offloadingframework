\chapter{Conclusion and Future Development}
\label{chapter:conclusion}

Mobile phone battery life has been the subject for debate among many enthusiasts and technicians alike, but not many have found a sustainable solution for long lasting smartphone capabilities. In this paper such a solution was proposed, through the use of the BLEOffloadingFramework.

Whereas most offloading frameworks currently being developed use the highly powerful cloud platforms in order to transfer computational power, this presents a limitation through the fact that it may take a long time for code or data to move from the mobile device to the cloud engine and back again.

In order to mitigate this limitation, the BLEOffloadingFramework uses Bluetooth Low Energy a communication channel between devices, thus mitigating the limitations of Internet connectivity and bringing offloading in a more local, well defined space. Besides the gain brought through offloading, this framework also presents a number of advantages to developers and users alike, such as:


\begin{itemize}

\item{It represents a new method of code offloading through a different communication channel.}

\item{It uses Bluetooth Low Energy, so that the overall energy consumption for Bluetooth Smart devices is heavily decreased.}

\item{Using this framework has shown a gain in battery life even for the simplest of applications.}

\item{It provides two different offloading methods and use cases for both of them.}

\item{It is simple to use and implement as a means of communicating data efficiently between devices.}

\item{It provides a test reference setup in order to present the actual gain the framework brings.}

\item{It is scalable for different machines and platforms.}

\item{It is transparent, providing notifications for the user whenever it will try a connection to an offload server.}

\end{itemize}

As such, the BLEOffloadingFramework presents to be a new method of doing offloading and even though the net gain is not as big as other available platforms or offloading systems, it represents a fresh way to use the power of Bluetooth Low Energy in order to augment already established concepts.

In the future, the BLEOffloadingFramework needs to implement even more features:

\begin{itemize}

\item{Portability - the framework should be available on other platforms as well, one main step would be to port the offloading server to a mobile device itself, in order to promote the concept of mobile distributed offloading, where devices can share code between them in order to achieve a more expensive task.}

\item{Security -  at the moment the framework relies on the encoding and decoding steps in order to provide security of data between devices, but this should change to a more secure channel and investigate the impact of using a more complex socket in the framework.}

\item{Efficiency - the framework requires a more efficient message passing system and should probably benefit more if this was changed completely to a GATT service characteristic reading/writing.}

\item{Content -  in order to provide developers with more and more options, content has to be added to the framework in terms of new tasks that can be delegate or third party applications that can be launched.}

\end{itemize}